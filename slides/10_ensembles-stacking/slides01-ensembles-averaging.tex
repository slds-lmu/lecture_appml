% OLD
%\documentclass[11pt,compress,t,notes=noshow, aspectratio=169, xcolor=table]{beamer}
% NEW
\documentclass[10pt,compress,t,notes=noshow, xcolor=table]{beamer}

\input{../../style/preamble}
\input{../../latex-math/basic-math}
\input{../../latex-math/basic-ml}
% ml - bagging, random forest
\newcommand{\bl}[1][m]{b^{[#1]}} % baselearner, default m
\newcommand{\blh}[1][m]{\hat{b}^{[#1]}} % estimated base learner, default m 
\newcommand{\blx}[1][m]{b^{[#1]}(\xv)} % baselearner, default m
\newcommand{\blxi}[1][m]{b^{[#1]}(\xv^{(i)})} % baselearner, default m
\newcommand{\fM}{\hat{f}^{[M]}} % ensembled predictor
\newcommand{\fMx}{\hat{f}^{[M]}(\xv)} % ensembled predictor
\newcommand{\fMh}{\hat{f}^{[M]}(\xv)} % estimated ensembled predictor
\newcommand{\ambifM}{\Delta\left(\fM\right)} % ambiguity/instability of ensemble
\newcommand{\betam}[1][m]{\beta^{[#1]}} % weight of basemodel m
\newcommand{\betamh}[1][m]{\hat{\beta}^{[#1]}} % weight of basemodel m with hat
\newcommand{\betaM}{\beta^{[M]}} % last baselearner
\newcommand{\hbl}[1][m]{b_{h}^{[#1]}} % baselearner hard label, default m
\newcommand{\sbl}[1][m]{b_{\pi}^{[#1]}} % baselearner probs, default m
\newcommand{\hblx}[1][m]{b_{h}^{[#1]}(\xv)} % baselearner hard label, default m
\newcommand{\sblx}[1][m]{b_{\pi}^{[#1]}(\xv)} % baselearner probs, default m
\newcommand{\fMw}{\hat{f}_{\mathbf{w}}^{[M]}} % ensembled predictor weighted
\newcommand{\fMwx}{\hat{f}_{\mathbf{w}}^{[M]}(\xv)} % ensembled predictor weighted
\newcommand{\fMwxi}{\hat{f}_{\mathbf{w}}^{[M]}(\xv^{(i)})} % ensembled predictor weighted
\newcommand{\w}{\mathbf{w}} % ensemble weight vector
\newcommand{\wstar}{\mathbf{w}^{*}} % optimal ensemble weight vector

% ml - boosting
%\newcommand{\fm}[1][m]{f^{[#1]}} % prediction in iteration m
%\newcommand{\fmh}[1][m]{\hat{f}^{[#1]}} % prediction in iteration m
%\newcommand{\fmd}[1][m]{f^{[#1-1]}} % prediction m-1
%\newcommand{\fmdh}[1][m]{\hat{f}^{[#1-1]}} % prediction m-1
%\newcommand{\errm}[1][m]{\text{err}^{[#1]}} % weighted in-sample misclassification rate
%\newcommand{\wm}[1][m]{w^{[#1]}} % weight vector of basemodel m
%\newcommand{\wmi}[1][m]{w^{[#1](i)}} % weight of obs i of basemodel m
%\newcommand{\thetam}[1][m]{\thetab^{[#1]}} % parameters of basemodel m
%\newcommand{\thetamh}[1][m]{\hat{\thetab}^{[#1]}} % parameters of basemodel m with hat
%\newcommand{\blxt}[1][m]{b(\xv, \thetab^{[#1]})} % baselearner, default m
%\newcommand{\ens}{\sum_{m=1}^M \betam \blxt} % ensemble
%\newcommand{\rmm}[1][m]{\tilde{r}^{[#1]}} % pseudo residuals
%\newcommand{\rmi}[1][m]{\tilde{r}^{[#1](i)}} % pseudo residuals
%\newcommand{\Rtm}[1][m]{R_{t}^{[#1]}} % terminal-region
%\newcommand{\Tm}[1][m]{T^{[#1]}} % terminal-region
%\newcommand{\ctm}[1][m]{c_t^{[#1]}} % mean, terminal-regions
%\newcommand{\ctmh}[1][m]{\hat{c}_t^{[#1]}} % mean, terminal-regions with hat
%\newcommand{\ctmt}[1][m]{\tilde{c}_t^{[#1]}} % mean, terminal-regions
%\newcommand{\Lp}{L^\prime}
%\newcommand{\Ldp}{L^{\prime\prime}}
%\newcommand{\Lpleft}{\Lp_{\text{left}}}

% ml - boosting iml lecture
%\newcommand{\ts}{\thetab^{\star}} % theta*
%\newcommand{\bljt}{\bl[j](\xv, \thetab)} % BL j with theta
%\newcommand{\bljts}{\bl[j](\xv, \ts)} % BL j with theta*
\input{../../latex-math/ml-eval}


\title{Applied Machine Learning}
% \author{LMU}
%\institute{\href{https://compstat-lmu.github.io/lecture_iml/}{compstat-lmu.github.io/lecture\_iml}}
\date{}

\begin{document}

\titlemeta{
Ensembling:
}{
Intro, Model Averaging
}{
figure/bagging_better.png
}{

\item What is Ensembling?
\item Bagging and Boosting Recap
\item Model Averaging

}


% FIXME: math and alphabets are totally messed up because we use way to many different fonts and styles in lectures

% ------------------------------------------------------------------------------

\begin{vbframe}{What is Ensembling?}

\vfill
\begin{columns}
    \begin{column}{0.5\textwidth}
        \vspace{-1em}
        \begin{figure}
            \centering
            \includegraphics[width=0.75\textwidth]{figure/Ensembling.pdf}
            %\caption{Caption}
            \label{fig:ensembling}
        \end{figure}
    \end{column}
    \begin{column}{0.5\textwidth}
        \begin{itemize}
            \item Combine the predictions of models
            \item Make weak learners strong
            \item Make strong learners stronger
            \item Also: Ensembling effectively \textit{circumvents} the problem of model selection
        \end{itemize}
    \end{column}
\end{columns}

\end{vbframe}

% ------------------------------------------------------------------------------

\begin{vbframe}{Recap: Bagging}

\textbf{B}ootstrap \textbf{Agg}regat\textbf{ing}: Training multiple instances of the same base model on different subsets of the training data, obtained through random sampling with replacement $\rightarrow$ homogeneous \& parallel ensemble
\vfill
\begin{columns}
    \begin{column}{0.6\textwidth}
        \vspace{-1em}
        \begin{figure}
            \centering
            % New version from 
            % https://docs.google.com/presentation/d/1gaeW1LefW3vQi3aoEiZEoSUaEyXfmQ9-U_1WoGKfGNM/edit#slide=id.g2e833da9e7c_0_143
            \includegraphics{figure/bagging_better.png}
            %\caption{Caption}
            % Old version from
            % from https://sebastianraschka.com/pdf/lecture-notes/stat479fs19/07_ensembles__notes.pdf
            \label{fig:bagging}
        \end{figure}
    \end{column}
    \begin{column}{0.4\textwidth}
        Steps:
        \begin{enumerate}
            \item Random sampling with replacement to generate different train data sets
            \item Base model training
            \item Combine predictions
        \end{enumerate}
    \end{column}
\end{columns}
\vfill
Popular use case: Construction of RF using trees as base models

\end{vbframe}

% ------------------------------------------------------------------------------

\begin{vbframe}{Recap: Bagging}
Bias vs. variance: Error = Variance + Bias + Noise
\begin{figure}
    \centering
    \includegraphics[width=0.5\textwidth]{figure_man/bagging_variance_bias_tree.png}
    %\caption{Caption}
    \label{fig:bagging_tree}
\end{figure}

\end{vbframe}

% ------------------------------------------------------------------------------

\begin{vbframe}{Recap: Bagging}
Bias vs. variance bagging: Variance reduction through averaging; Bias of model is slightly increased as less data are seen\\
\begin{columns}
    \begin{column}{0.5\textwidth}
        \vspace{-1em}
        \begin{figure}
            \centering
            \includegraphics{figure_man/bagging_variance_bias_tree.png}
            %\caption{Caption}
            \label{fig:bagging_tree2}
        \end{figure}
    \end{column}
    \begin{column}{0.5\textwidth}
        \vspace{-1em}
        \begin{figure}
            \centering
            \includegraphics{figure_man/bagging_variance_bias_100_trees.png}
            %\caption{Caption}
            \label{fig:bagging_trees}
        \end{figure}
    \end{column}
\end{columns}
Works well for unstable learners\\
\vfill
NB: RF also uses col subsampling

\end{vbframe}

% ------------------------------------------------------------------------------

\begin{vbframe}{Recap: Boosting}
Iterative technique that sequentially trains subsequent base models to correct errors of previous models, for example, by re-weighting instances or predicting pseudo-residuals.
\vfill
Examples:
\begin{itemize}
    \item AdaBoost (Adaptive Boosting)
    \item Gradient Boosting
    \item XGBoost (Extreme Gradient Boosting)
    \item LightGBM (fast and memory efficient Gradient Boosting)
    \item CatBoost (native support for categorical features + ordered boosting)
    \item Model Based Boosting
\end{itemize}
\vfill
\end{vbframe}

% ------------------------------------------------------------------------------

\begin{vbframe}{General Notation for Ensembling}
\vfill
\begin{itemize}
    \item $\{\bl[1], \ldots, \bl[M]\} \subseteq \mathcal{H}_{1} \times \ldots \times \mathcal{H}_{M}$ set of $M$ models trained on training data $\D$
    \item $\sbl[m]: \Xspace \rightarrow [0, 1]^{g}$ probability predictions under the $m$-th model when doing classification (w.l.o.g. we assume $g = 1$ in the following and $\sbl[m]$ returning the probabilities of the positive class)
    \item $\hbl[m]: \Xspace \rightarrow \Yspace$ hard label predictions under the $m$-th model when doing classification or regression
    \item $\fMx = G(\blx[1], \ldots, \blx[M])$ prediction of an ensemble model obtained via aggregating function $G$
\end{itemize}
\vfill
\end{vbframe}

% ------------------------------------------------------------------------------
\begin{vbframe}{Model Averaging}
\vfill
Combine predictions of multiple (different) models to make a final prediction $\rightarrow$ heterogeneous
\vfill
\begin{columns}
    \begin{column}{0.6\textwidth}
        \vspace{-1em}
        \begin{figure}
            \centering
            % New version from
            % https://docs.google.com/presentation/d/1gaeW1LefW3vQi3aoEiZEoSUaEyXfmQ9-U_1WoGKfGNM/edit#slide=id.g2e833da9e7c_0_143
            \includegraphics[width=0.7\textwidth]{figure/voting_better.png}
            %\caption{Caption}
            % Old version from
            % from https://sebastianraschka.com/pdf/lecture-notes/stat479fs19/07_ensembles__notes.pdf
            \label{fig:voting}
        \end{figure}
    \end{column}
    \begin{column}{0.4\textwidth}
        Steps:
        \begin{enumerate}
            \item Base model training
            \item Combine predictions
        \end{enumerate}
    \end{column}
\end{columns}
\vfill
Classification: majority voting (hard voting) / model averaging (soft voting)
\begin{itemize}
    \item $\fMx = \text{mode}(\{\hblx[1], \ldots, \hblx[M]\})$ assuming hard label predictions
    \item $\fMx = \frac{1}{M} \sum_{m = 1}^{M} \sblx[m]$ assuming probability predictions
\end{itemize}
\vfill
Regression: model averaging
\begin{itemize}
    \item $\fMx = \frac{1}{M} \sum_{m = 1}^{M} \hblx[m]$
\end{itemize}
\vfill
\end{vbframe}

% ------------------------------------------------------------------------------

\begin{vbframe}{Weighted Model Averaging}
Assume a set of $M$ different models obtained from data $\D$ using a resampling $\JJ$.\\
\vfill
Assign weights $\w = (w_{1}, \ldots, w_{M}), \text{usually: } 0 \le w_{m} \le 1, \sum_{m=1}^{M} w_{m} = 1$ to the models\\
\vfill
$\fMwx = \frac{1}{M} \sum_{m = 1}^{M} w_{m} \cdot \sblx[m]$\\
\begin{figure}
    \centering
    \includegraphics[width=0.375\textwidth]{figure/Weighted_Model_Averaging.pdf}
    %\caption{Caption}
    \label{fig:weighted_model_averaging}
\end{figure}

\end{vbframe}

% ------------------------------------------------------------------------------

\begin{vbframe}{Example: Weighted Model Averaging}
% rsrc/ensembles_averaging.R
\vfill
Balanced accuracy (BACC) on the sonar task, using a stratified holdout split (1/3 test set):
\vfill
\begin{table}[h]
\centering
\begin{tabular}{lc}
\hline
\textbf{Learner} & \textbf{BACC} (Test) \\
\hline
RF & 0.834 \\
k-NN (k=7) & 0.792 \\
Log. Reg. & 0.695 \\
Model Averaging (Equal Weights) & 0.796 \\
Model Averaging ($\w = (0.75, 0.05, 0.20)^\intercal)$ & 0.810 \\
\hline
\end{tabular}
\label{table:learners_test}
\end{table}
\vfill
How can we find 'good' weights?\\
\vfill
NB: Model averaging with equal weights will not result in better performance out of the box; depends on types of learners, their diversity in predictions, ...
\vfill
\end{vbframe}

\begin{frame}{Weighted Model Averaging Continued}
Find the optimal weight vector $\wstar$ such that the estimated generalization error (based on a performance metric $\rho$ or loss function $L$) of the ensemble is optimal:
$$\wstar \coloneqq \operatorname*{arg\,min}_{\w \in \mathbb{R}^{M}} \hat{\text{GE}}\left(\hat{f}_{\mathbf{w}}, \JJ, \rho\right)$$
\vfill
Performance metric/loss function unknown a priori (e.g., as in AutoML) $\rightarrow$ black box optimization problem.
\vfill
Difficulty of the optimization problem depends on the metric/loss function:
\begin{itemize}
    \item MSE (regression) $L_{\mathrm{MSE}}(y^{(i)}, \hat{f}_{\mathbf{w}}(\mathbf{x}^{(i)})) = (y^{(i)} - \hat{f}_{\mathbf{w}}(\mathbf{x}^{(i)}))^{2}$;\\
          Constraints of the form $0 \le w_{m} \le 1, \sum_{m=1}^{M} w_{m} = 1$;\\
          $$\operatorname*{arg\,min}_{\w \in \mathbb{R}^{M}} \hat{\text{GE}}(\hat{f}_{\mathbf{w}}, \JJ, L_{\mathrm{MSE}}) \text{ s.t. } 0 \le w_{m} \le 1, \sum_{m=1}^{M} w_{m} = 1$$ 
          can be written as a quadratic program.
    \item 0-1 loss (binary classification); the optimization problem is NP-hard.
\end{itemize}
\vfill
In practice, we use black box optimization, greedy ensemble selection, or a linear model (stacking).
\end{frame}


% ------------------------------------------------------------------------------

\begin{vbframe}{Greedy Ensemble Selection}
Assume a set of $M$ different models obtained from data $\D$ using a resampling $\JJ$.\\
We now want to optimize the weights used in weighted model averaging.\\
\vfill
Greedy ensemble selection (GES, see \furtherreading{CARUANA2004}) is one of the most popular methods to do so:\\\
\begin{enumerate}
    \item Start with the empty ensemble
    \item Add the base model that maximizes the ensembles' performance metric on a hillclimb (validation) set
    \item Repeat step 2 until a given number of iterations are reached
    \item Return the ensemble from the set of generated ensembles that has the best performance on the validation set
\end{enumerate}
\vfill
NB: GES optimizes the weights on an integer domain (i.e., base models are selected one or multiple times or not at all\footnote{This has the (positive) side-effect of resulting in a sparse weight vector}.)
\end{vbframe}

%------------------------------------------------------------------------------

\begin{vbframe}{Example: GES}
\begin{table}[h]
\footnotesize
\centering
\begin{tabular}{lccccc}
\hline
\textbf{Iter} & \textbf{Sel. Learners} & \textbf{Current BACC} (Valid) & \multicolumn{3}{c}{\textbf{BACC if Learner added} (Valid)}\\ \cline{4-6}
& & & RF & k-NN & Log. Reg.\\ \hline
1  & $\{$RF: 0, k-NN: 0, Log. Reg.: 0$\}$ & NA    & \textbf{0.837} & 0.809 & 0.766\\
2  & $\{$RF: 1, k-NN: 0, Log. Reg.: 0$\}$ & 0.837 & \textbf{0.837} & 0.809 & 0.766\\
3  & $\{$RF: 2, k-NN: 0, Log. Reg.: 0$\}$ & 0.837 & \textbf{0.837} & 0.809 & 0.809\\
4  & $\{$RF: 3, k-NN: 0, Log. Reg.: 0$\}$ & 0.837 & \textbf{0.837} & 0.769 & 0.832\\
5  & $\{$RF: 4, k-NN: 0, Log. Reg.: 0$\}$ & 0.837 & 0.837 & 0.792 & \textbf{0.855}\\
6  & $\{$RF: 4, k-NN: 0, Log. Reg.: 1$\}$ & 0.855 & 0.897 & \textbf{0.917} & 0.809\\
7  & $\{$RF: 4, k-NN: 1, Log. Reg.: 1$\}$ & \textbf{0.917} & \textbf{0.917} & 0.895 & 0.809\\
8  & $\{$RF: 5, k-NN: 1, Log. Reg.: 1$\}$ & 0.917 & \textbf{0.897} & 0.875 & 0.875\\
9  & $\{$RF: 6, k-NN: 1, Log. Reg.: 1$\}$ & 0.897 & \textbf{0.877} & 0.875 & 0.875\\
10 & $\{$RF: 7, k-NN: 1, Log. Reg.: 1$\}$ & 0.877 & 0.857 & 0.855 & \textbf{0.917}\\
11 & $\{$RF: 7, k-NN: 1, Log. Reg.: 2$\}$ & 0.917 & ... & ... & ...\\
\hline
\end{tabular}
\label{table:learners_valid_test}
\end{table}
Best iteration: 6 with BACC (Valid) of $0.917$\\
$\rightarrow$ $\{$RF: 4, k-NN: 1, Log. Reg.: 1$\}$\\
$\rightarrow$ $\wstar = (\frac{4}{6}, \frac{1}{6}, \frac{1}{6})^\intercal$
\end{vbframe}

%------------------------------------------------------------------------------

\begin{vbframe}{Example: Conclusion}
\begin{table}[h]
\centering
\begin{tabular}{lcc}
\hline
\textbf{Learner} & \textbf{BACC} (Valid) & \textbf{BACC} (Test) \\
\hline
RF & 0.837 & 0.834 \\
k-NN (k=7) & 0.809 & 0.792 \\
Log. Reg. & 0.766 & 0.695 \\
Model Averaging (Equal Weights) & 0.832 & 0.796 \\
Model Averaging ($\w = (0.75, 0.05, 0.20)^\intercal)$ & 0.875 & 0.810 \\
Model Averaging (GES: $\wstar = (\frac{4}{6}, \frac{1}{6}, \frac{1}{6})^\intercal$) & 0.917 & 0.864\\
\hline
\end{tabular}
\label{table:learners_valid_test_final}
\end{table}
\vfill
NB: Optimizing weights can (and often does) overfit to the validation data; especially if standard black box optimizers are used\\
\end{vbframe}

%------------------------------------------------------------------------------

\begin{vbframe}{Closing}
\vfill
Why does ensembling work? \furtherreading{DIETTERICH2000}
\begin{itemize}
    \item Statistical Problem
    \item Computational Problem
    \item Representational Problem
\end{itemize}
\vfill
Requirements to build an ensemble:
\begin{itemize}
    \item Diverse Models
    \item Strong Models
\end{itemize}
\vfill
Practical Suggestion: Greedy Ensemble Selection works really well in practice.
\vfill
\end{vbframe}

% ------------------------------------------------------------------------------


\endlecture
\end{document}