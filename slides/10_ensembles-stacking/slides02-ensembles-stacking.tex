\documentclass[11pt,compress,t,notes=noshow, xcolor=table]{beamer}

\input{../../style/preamble}
\input{../../latex-math/basic-math}
\input{../../latex-math/basic-ml}
% ml - bagging, random forest
\newcommand{\bl}[1][m]{b^{[#1]}} % baselearner, default m
\newcommand{\blh}[1][m]{\hat{b}^{[#1]}} % estimated base learner, default m 
\newcommand{\blx}[1][m]{b^{[#1]}(\xv)} % baselearner, default m
\newcommand{\blxi}[1][m]{b^{[#1]}(\xv^{(i)})} % baselearner, default m
\newcommand{\fM}{\hat{f}^{[M]}} % ensembled predictor
\newcommand{\fMx}{\hat{f}^{[M]}(\xv)} % ensembled predictor
\newcommand{\fMh}{\hat{f}^{[M]}(\xv)} % estimated ensembled predictor
\newcommand{\ambifM}{\Delta\left(\fM\right)} % ambiguity/instability of ensemble
\newcommand{\betam}[1][m]{\beta^{[#1]}} % weight of basemodel m
\newcommand{\betamh}[1][m]{\hat{\beta}^{[#1]}} % weight of basemodel m with hat
\newcommand{\betaM}{\beta^{[M]}} % last baselearner
\newcommand{\hbl}[1][m]{b_{h}^{[#1]}} % baselearner hard label, default m
\newcommand{\sbl}[1][m]{b_{\pi}^{[#1]}} % baselearner probs, default m
\newcommand{\hblx}[1][m]{b_{h}^{[#1]}(\xv)} % baselearner hard label, default m
\newcommand{\sblx}[1][m]{b_{\pi}^{[#1]}(\xv)} % baselearner probs, default m
\newcommand{\fMw}{\hat{f}_{\mathbf{w}}^{[M]}} % ensembled predictor weighted
\newcommand{\fMwx}{\hat{f}_{\mathbf{w}}^{[M]}(\xv)} % ensembled predictor weighted
\newcommand{\fMwxi}{\hat{f}_{\mathbf{w}}^{[M]}(\xv^{(i)})} % ensembled predictor weighted
\newcommand{\w}{\mathbf{w}} % ensemble weight vector
\newcommand{\wstar}{\mathbf{w}^{*}} % optimal ensemble weight vector

% ml - boosting
%\newcommand{\fm}[1][m]{f^{[#1]}} % prediction in iteration m
%\newcommand{\fmh}[1][m]{\hat{f}^{[#1]}} % prediction in iteration m
%\newcommand{\fmd}[1][m]{f^{[#1-1]}} % prediction m-1
%\newcommand{\fmdh}[1][m]{\hat{f}^{[#1-1]}} % prediction m-1
%\newcommand{\errm}[1][m]{\text{err}^{[#1]}} % weighted in-sample misclassification rate
%\newcommand{\wm}[1][m]{w^{[#1]}} % weight vector of basemodel m
%\newcommand{\wmi}[1][m]{w^{[#1](i)}} % weight of obs i of basemodel m
%\newcommand{\thetam}[1][m]{\thetab^{[#1]}} % parameters of basemodel m
%\newcommand{\thetamh}[1][m]{\hat{\thetab}^{[#1]}} % parameters of basemodel m with hat
%\newcommand{\blxt}[1][m]{b(\xv, \thetab^{[#1]})} % baselearner, default m
%\newcommand{\ens}{\sum_{m=1}^M \betam \blxt} % ensemble
%\newcommand{\rmm}[1][m]{\tilde{r}^{[#1]}} % pseudo residuals
%\newcommand{\rmi}[1][m]{\tilde{r}^{[#1](i)}} % pseudo residuals
%\newcommand{\Rtm}[1][m]{R_{t}^{[#1]}} % terminal-region
%\newcommand{\Tm}[1][m]{T^{[#1]}} % terminal-region
%\newcommand{\ctm}[1][m]{c_t^{[#1]}} % mean, terminal-regions
%\newcommand{\ctmh}[1][m]{\hat{c}_t^{[#1]}} % mean, terminal-regions with hat
%\newcommand{\ctmt}[1][m]{\tilde{c}_t^{[#1]}} % mean, terminal-regions
%\newcommand{\Lp}{L^\prime}
%\newcommand{\Ldp}{L^{\prime\prime}}
%\newcommand{\Lpleft}{\Lp_{\text{left}}}

% ml - boosting iml lecture
%\newcommand{\ts}{\thetab^{\star}} % theta*
%\newcommand{\bljt}{\bl[j](\xv, \thetab)} % BL j with theta
%\newcommand{\bljts}{\bl[j](\xv, \ts)} % BL j with theta*
\input{../../latex-math/ml-eval}

% algorithmicx (algpseudocode) is already loaded in preamble


\title{Applied Machine Learning}
% \author{LMU}
%\institute{\href{https://compstat-lmu.github.io/lecture_iml/}{compstat-lmu.github.io/lecture\_iml}}
\date{}

\begin{document}

\titlemeta{
Ensembling:
}{
Stacking
}{
figure/stacking_better.png
}{
\item What is Stacking?
\item How can we stack multiple layers?
}


\begin{vbframe}{Stacking}
Also called stacked generalization (\furtherreading{WOLPERT1992}): train multiple base models (level 0) on the same data and use their predictions as input features for a level 1 model that makes the final ensemble prediction $\rightarrow$ heterogeneous ensembling
\vspace{1em}
\begin{columns}
    \begin{column}{0.5\textwidth}
        \vspace{-1em}
        \begin{figure}
            \centering
            % New version from
            % https://docs.google.com/presentation/d/1gaeW1LefW3vQi3aoEiZEoSUaEyXfmQ9-U_1WoGKfGNM/edit#slide=id.g2e833da9e7c_0_143
            \includegraphics[width=0.9\textwidth]{figure/stacking_better.png}
            %\caption{Caption}
            % Old version from 
            % from https://sebastianraschka.com/pdf/lecture-notes/stat479fs19/07-ensembles__notes.pdf
            \label{fig:stacking}
        \end{figure}
    \end{column}
    \begin{column}{0.5\textwidth}
        Steps:
        \begin{enumerate}
            \item Base model training (level 0)
            \item Collect predictions
            \item level 1 model training
        \end{enumerate}
    \end{column}
\end{columns}
\vspace{1em}

In contrast to model averaging, stacking already combines predictions during training.

\end{vbframe}

% ------------------------------------------------------------------------------

\begin{vbframe}{Stacking}
\footnotesize
\textbf{Algorithm:} Single Layer Stacking
\begin{algorithmic}
    \Require Training data $\D = \Dset$
    \Ensure Stacked Ensemble $\fM$
    
    \State
    \State \textbf{Step 1: Learn level 0 base models}
    \For{$m \leftarrow 1$ to $M$}
        \State Fit base model $\bl[m]$ on $\D$
    \EndFor
    
    \State
    \State \textbf{Step 2: Construct new dataset $\D^\prime$ from $\D$}
    \For{$i \leftarrow 1$ to $n$}
        \State Add $(\xv^{\prime(i)}, y^{(i)})$ to new dataset, where $\xv^{\prime(i)} = (\bl[1](\xv^{(i)}), \ldots, \bl[M](\xv^{(i)}))$
    \EndFor
    
    \State
    \State \textbf{Step 3: Fit level 1 model $\fM$ on $\D^\prime$}
    
    \State \Return $\fM$
\end{algorithmic}
NB: We can either drop the original features or keep them when learning the level 1 model.

\end{vbframe}

% ------------------------------------------------------------------------------

\begin{vbframe}{Stacking Cross-validated}

\begin{columns}
    \begin{column}{0.6\textwidth}
        \vspace{-1em}
        \begin{figure}
            \centering
            % New version from 
            % https://docs.google.com/presentation/d/1gaeW1LefW3vQi3aoEiZEoSUaEyXfmQ9-U_1WoGKfGNM/edit#slide=id.g2e833da9e7c_0_143
            \includegraphics[width=0.7\textwidth]{figure/stacking_cv_better.png}
            % Old version from
            %\caption{Caption}
            % from https://sebastianraschka.com/pdf/lecture-notes/stat479fs19/07-ensembles__notes.pdf
            \label{fig:stacking_cv}
        \end{figure}
    \end{column}
    \begin{column}{0.4\textwidth}
        Using cross-validated predictions of the base models
        \begin{itemize}
            \item improve generalization
            \item reduce overfitting
            \item enhances base model diversity
            \item minimizes information leakage
        \end{itemize}
        \vspace{1em}
        NB: Stacking with a simple hold-out split instead of cross-validation is known as \emph{blending}.
    \end{column}
\end{columns}
\end{vbframe}

% ------------------------------------------------------------------------------

\begin{vbframe}{Stacking Cross-validated}
\footnotesize
\textbf{Algorithm:} Single Layer Stacking with Cross-Validation
\begin{algorithmic}
    \Require Training data $\D = \Dset$, number of folds $K$
    \Ensure Stacked Ensemble $\fM$
    
    \For{$k \leftarrow 1$ to $K$}
        \State \textbf{Step 1: Learn level 0 base models}
        \For{$m \leftarrow 1$ to $M$}
            \State Fit base model $\bl[m]$ on $\D_{\mathrm{train}}^{(k)}$
        \EndFor
        \State
        \State \textbf{Step 2: Add out-of-fold predictions to new dataset $\D^\prime$}
        \ForAll{$(\xv^{(i)}, y^{(i)}) \in \D_{\mathrm{test}}^{(k)}$}
            \State Add $(\xv^{\prime(i)}, y^{(i)})$ to new dataset, where $\xv^{\prime(i)} = (\blxi[1], \ldots, \blxi[M])$
        \EndFor
    \EndFor

    \State
    \State \textbf{Step 3: Fit level 1 model $\fM$ on $\D^\prime$}
    
    \State \Return $\fM$
\end{algorithmic}

\end{vbframe}

% ------------------------------------------------------------------------------

\begin{vbframe}{Multilevel Stacking}
\vfill
So far: train level 0 base models, use predictions to train the level 1 model.\\
Now: Also train multiple models at level 1, use predictions to train final level 2 model 
\vfill
%\begin{columns}
%    \begin{column}{0.5\textwidth}
%        \vspace{-1em}
%        \begin{figure}
%            \centering
%            \includegraphics[width=0.9\textwidth]{figure/stacking.png}
%            %\caption{Caption}
%            % from https://sebastianraschka.com/pdf/lecture-notes/stat479fs19/07-ensembles__notes.pdf
%            \label{fig:stacking_multi}
%        \end{figure}
%    \end{column}
%    \begin{column}{0.5\textwidth}
%        Steps:
%        \begin{enumerate}
%            \item level 0 base model training
%            \item Collect predictions of level 0 base models
%            \item level 1 model training 
%            \item Collect predictions of level 1 models
%            \item level 2 final model training
%        \end{enumerate}
%    \end{column}
%\end{columns}
\begin{enumerate}
    \item level 0 base model training
    \item Collect predictions of level 0 base models
    \item level 1 model training 
    \item Collect predictions of level 1 models
    \item level 2 final model training
\end{enumerate}
\vfill
NB: In principle we can repeat this training of multiple models at the $l$-th level indefinitely creating deep stacked ensembles.
\vfill
\end{vbframe}

% ------------------------------------------------------------------------------

\begin{vbframe}{Higher Level Models}
\vfill
Popular choices of level 1 or level 2 models in stacking include:
\begin{itemize}
    \item linear / logistic regression 
    %\item random forest (especially in the case of multilevel stacking)
    \item GES (average the predictions of the models at the previous level)
\end{itemize}
\vfill
Higher-level training data:
\begin{itemize}
    \item Probabilities instead of class labels (see \furtherreading{TINGWITTEN1999})
    \item Include original dataset $\Xmat$ or not?
\end{itemize}
\vfill
\end{vbframe}

% ------------------------------------------------------------------------------

\begin{vbframe}{Forward Pass / Prediction}
% \vspace{-1.5em}
\textbf{Algorithm:} Prediction Using Single Layer Stacking
\begin{algorithmic}
    \State
    \Require Training data $\D = \Dset$, Test data $\Xmat$
    \Ensure Predictions $\yv$

    \State \textbf{Step 1: Fit level-0 on the whole training data set}
    \For{$m \leftarrow 1$ to $M$}
        \State Fit level-0 model $\bl[m]$ on $\D$
    \EndFor
    
    \State
    \State \textbf{Step 2: Make predictions with every level-0 model}
    \For{$m \leftarrow 1$ to $M$}
        \State Predict with base model: $\yv^m = \sbl[m](\Xmat)$
    \EndFor
    
    \State
    \State \textbf{Step 3: Construct new dataset $\Xmat^\prime$}
    \For{$i \leftarrow 1$ to $n$}
        \State $\Xmat^{\prime{i}} = [\xi[i]_1, \dots, \xi[i]_d, \xv^{\prime{(i)}}_1, \dots, \xv^{\prime{(i)}}_M] $
    \EndFor
    
    \State
    \State \textbf{Step 4: Make predictions with level 1 model $\fM$ on $\Xmat^\prime$}
    
    \State \Return $\fM(\Xmat)$
\end{algorithmic}

\end{vbframe}

% ------------------------------------------------------------------------------

\begin{vbframe}{Two questions}

\vfill
What other methods do the following level-1 classifiers recover?
\vfill
\begin{itemize}
    \item Fit the level-1 training data with a linear model that only depends on one feature, and that feature is chosen such that $\rho$, our metrics of interest, is optimized.
    \item The level-1 model is a linear model that assign equal weight $w_{m} = \frac{1}{M}$ to every input feature.
\end{itemize}
\vfill
    
\end{vbframe}

% ------------------------------------------------------------------------------

\begin{vbframe}{AutoGluon}
\begin{columns}
    \begin{column}{0.4\textwidth}
        \vspace{-1em}
        \begin{figure}
            \centering
            \includegraphics[width=\textwidth]{figure/AG.png}
            \caption*{\url{https://auto.gluon.ai/}}
            \label{fig:auto_gluon}
        \end{figure}
        \vspace{1em}
        \begin{figure}
            \centering
            \includegraphics[width=\textwidth]{figure/ALMB.png}
            \caption{AutoGluon Performance on AMLB 1h binary classification.}
            \label{fig:auto_gluon_almb}
        \end{figure}
    \end{column}
    \begin{column}{0.6\textwidth}
        \begin{itemize}
            \item The de-facto most popular and successful AutoML system
            \item Developed and maintained by Amazon
            \item Can also work with multimodal and time-series data
            \item Does not use HPO to find well performing models
            \begin{itemize}
                \item Instead, uses a set of default configurations for different learners
            \end{itemize}
            \item Employs multi-layer stacking using the same set of default models in higher layers as for the base layer
            \item Uses greedy ensemble selection to combine the predictions of the last layer
        \end{itemize}
    \end{column}
\end{columns}
\end{vbframe}

% ------------------------------------------------------------------------------

%\begin{vbframe}{Further Reading}
%
%\end{vbframe}

% ------------------------------------------------------------------------------

\endlecture
\end{document}
