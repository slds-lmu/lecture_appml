% Target Transformation

\begin{vbframe}{Target Transformation}

Sometimes, using the raw target or features is not enough to build an adequate model.
For example, the linear model requires a normally distributed target variable.
But the house prices do not seems to be normal distributed.
A linear model trained on that target overestimates the target variable:

\vfill

\begin{center}
\includegraphics[width = 0.7\textwidth]{figure/ames_target_distribution.pdf}
\end{center}

\vfill

\end{vbframe}

% Target Transformation - Log transformation

\begin{vbframe}{Target Transformation}

A common trick for skewed distributions is to model the log-transformation:

\vfill

\begin{center}
\includegraphics[width = 0.7\textwidth]{figure/ames_log_target_distribution.pdf}
\end{center}

\vfill

\end{vbframe}

% Target Transformation - Benchmarking

\begin{vbframe}{Target Transformation}

Benchmarking the logarithmic transformation against the raw data yields a significant improvement of the mean absolute error:

\vfill

\begin{center}
\includegraphics[width = 0.7\textwidth]{figure/target_transformation_benchmark.pdf}
\end{center}

\vfill

\end{vbframe}

% Target Transformation - Other methods

\begin{vbframe}{Target Transformation}

Nevertheless, there are also methods that are able to deal with skewed data:

\vfill

\begin{center}
\includegraphics[width = 0.7\textwidth]{figure/transformation_comparison_algorithms.pdf}
\end{center}

\vfill

\end{vbframe}
