\input{../../style/preamble}
\input{../../latex-math/basic-math}
\input{../../latex-math/basic-ml}

\newcommand{\titlefigure}{ames_dataset_title.png}
\newcommand{\learninggoals}{
  \item Understand the Ames Housing dataset structure and complexity
  \item Learn about different feature types in real-world data
  \item Recognize data preprocessing challenges
  \item Appreciate dataset complexity beyond toy examples
}

\title{Feature Engineering: Ames Housing Dataset}
\date{}

\begin{document}

\lecturechapter{Ames Housing Dataset}
\lecture{Applied Machine Learning}

% Set style/preamble.tex as framework

\sloppy

% ------------------------------------------------------------------------------
% LECTURE CONTENT
% ------------------------------------------------------------------------------

% Ames Housing Data Set

\begin{vbframe}{Ames Housing Data Set}

Ames is a small city in Iowa, USA.

It describes the sale of individual residential properties from 2006 to 2010.

The data was collected by Dean De Cock as an more complex alternative to the often used Boston Housing dataset.

It contains 2930 observations and

\begin{itemize}
\item 23 categorical features,
\item 23 ordered categorical features,
\item 12 integer features,
\item 20 continuous features,
\item 1 functional feature describing power usage of houses over a day.
\end{itemize}

\textbf{Note:} This is a slightly changed dataset from the original one.

\end{vbframe}

% Ames Housing Data Set - Target

\begin{vbframe}{Ames Housing Data Set - Target}

The goal is to predict the selling price based on these features

\vfill

\begin{center}
\includegraphics[width = 0.8\textwidth]{figure/ames_saleprice_histogram.pdf}
\end{center}

\vfill

\end{vbframe}

\endlecture
\end{document}
