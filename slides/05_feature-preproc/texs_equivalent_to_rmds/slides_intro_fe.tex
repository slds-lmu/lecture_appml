% Machine Learning Workflow

\begin{vbframe}{Machine Learning Workflow}

\vfill

\begin{center}
\includegraphics[width = 0.75\textwidth]{figure/ml-workflow-big.pdf}
\end{center}

\vfill

\end{vbframe}

% Machine Learning Pipelines

\begin{vbframe}{Machine Learning Pipelines}

\vfill

\begin{center}
\includegraphics[width = 0.75\textwidth]{figure/automl2.pdf}
\end{center}

\vfill

Choose pipeline structure and optimize pipeline parameters w.r.t. the estimated prediction error, on an independent test set, or measured by cross-validation.

\end{vbframe}

% Important types of Feature Engineering

\begin{vbframe}{Important types of Feature Engineering}

Feature engineering is on the intersection of \textbf{data cleaning}, \textbf{feature creation} and \textbf{feature selection}.

\lz

The goal is to solve common difficulties in data science projects, like

\begin{itemize}
\item skewed/\textit{weird} feature distributions,
\item (high cardinality) categorical features,
\item functional (temporal) features,
\item missing observations,
\item high dimensional data,
\item ...
\end{itemize}

and \textbf{improve model performance}.

\end{vbframe}

% Why Feature Engineering is Important

\begin{vbframe}{Why Feature Engineering is Important}

\begin{center}
\includegraphics[width = 0.8\textwidth]{figure/feature_engineering_importance.pdf}
\end{center}

Choice between a simple \textbf{interpretable} model with feature engineering or a complex model without.

\end{vbframe}

% Feature Engineering and Deep Learning

\begin{vbframe}{Feature Engineering and Deep Learning}

One argument for deep learning is often the idea of \textit{"automatic feature engineering"}, i.e., that no further preprocessing steps are necessary.

\lz

\textbf{This is mainly true for special types of data like}

\begin{itemize}
\item Images
\item Texts
\item Curves/Sequences
\end{itemize}

Many feature engineering problems for regular \textbf{tabular} data are not solved by deep learning.

Furthermore, choosing the architecture and learning hyperparameters poses its own new challenges.

\end{vbframe}
