\documentclass[11pt,compress,t,notes=noshow, xcolor=table]{beamer}

\input{../../style/preamble}
\input{../../latex-math/basic-math}
\input{../../latex-math/basic-ml}


\title{Applied Machine Learning}
% \author{LMU}
%\institute{\href{https://compstat-lmu.github.io/lecture_iml/}{compstat-lmu.github.io/lecture\_iml}}
\date{}

\begin{document} 

\titlemeta{
Imputation
}{
Model-Based Methods
}{
figure_man/fe_imputation_models_first_slide
}{

\item Model-based imputation approach
\item Using feature correlations for imputation
\item Drawbacks of model-based methods
\item Choice of surrogate models
\item Handling missing values in surrogate models

}

\begin{frame}{Model-Based Imputation}

    Instead of imputing a single value or sampling values it is desirable to take advantage of structure and correlation between features.
    
    \begin{center}
        \includegraphics[width=\textwidth]{figure_man/fe_imputation_models_first_slide}
    \end{center}

\end{frame}

\begin{frame}{Model-Based Imputation: Drawbacks}

    \begin{itemize}
        \item Choice of surrogate model has high influence on the imputation:
    \end{itemize}
    
    \begin{center}
        \includegraphics[width=0.4\textwidth]{figure/surrogate_model_influence}
    \end{center}
    
    \begin{itemize}
        \item Surrogate model should handle missing values itself, otherwise imputation \textit{loop} may be necessary.
        
        \item Surrogate model hyperparameters can be tuned and can be different for each feature to impute.
    \end{itemize}

\end{frame}

\endlecture
\end{document}
