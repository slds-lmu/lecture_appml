\documentclass[11pt,compress,t,notes=noshow, xcolor=table]{beamer}

\input{../../style/preamble}
\input{../../latex-math/basic-math}
\input{../../latex-math/basic-ml}


\title{Applied Machine Learning}
% \author{LMU}
%\institute{\href{https://compstat-lmu.github.io/lecture_iml/}{compstat-lmu.github.io/lecture\_iml}}
\date{}

\begin{document} 

\titlemeta{
Imputation:
}{
Introduction
}{
figure/missing_values_plot
}{
\item Motivating example
\item Visualizing missing values
\item Approaches to handle missing data
}

\begin{frame}{Motivating Example}

    \begin{itemize}
        \item Assume each feature in your dataset has 2\% missing values.
        \item The missing values are randomly distributed over the observations.
        \item How many rows can be used if all observations that contain at least a missing value is dropped?
    \end{itemize}
    
    \begin{center}
        \includegraphics[width=0.6\textwidth]{figure/missing_values_plot}
    \end{center}
    
    With 100 features and 2\% missing values only 13\% of our data can be used.

\end{frame}

\begin{frame}{Visualizing Missing Values}

    \begin{center}
        \includegraphics[width=\textwidth]{figure/missing_values_visualization}
    \end{center}

\end{frame}

\begin{frame}{Possible Ways to Deal With Missing Values}

    \begin{itemize}
        \item Remove observations that contain missing values. \\
              \textbf{But:} Could lead to a very small dataset.
        
        \item Remove features that contain mostly missing values. \\
              \textbf{But:} Can lose (important) information.
        
        \item Use models that can handle missing values, e.g., (most) tree-based methods \\
              \textbf{But:} Restriction in model choice.
        
        \item \textbf{Imputation} \\
              $\rightarrow$ Replace missing values with \textit{plausible} values.
    \end{itemize}

\end{frame}

\endlecture
\end{document}
